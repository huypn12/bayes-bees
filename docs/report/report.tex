\documentclass[12pt]{article}

\usepackage[utf8]{inputenc}

% graphics
\usepackage{graphicx}
\usepackage[pdf]{graphviz}

% math packages
\usepackage{amsmath}
\usepackage{amsfonts}
\usepackage{amssymb}
\usepackage{amsthm}

% citation
\usepackage{biblatex}
\addbibresource{progress_1.bib}

% MACROS
\newcommand*{\QED}{\hfill\ensuremath{\square}}
\newcommand{\stirlingii}{\genfrac{\{}{\}}{0pt}{}}

% title
\title{Master Project on Modeling of Complex, Self-organizing systems\\Bayesian
  Parameter Synthesis of Discrete Time Markov Chains}
\author{Huy Phung\\University of Konstanz}
\date{\today}

\setlength{\headheight}{23pt}
\setlength{\parindent}{0.0in}
\setlength{\parskip}{0.0in}

\begin{document}
\maketitle

\section{Abstract}
The aim of this project is to use Bayesian method for parameter synthesis of
Discrete Time Markov Chain (DTMC). Bayesian inference 



\section{Preliminaries}
\subsection{Discrete time Markov Chain}
Markov process

Discrete Time Markov Chain

Definition and parameter inference problem is defined as
\cite{katoen2016probabilistic}


\subsection{Bayesian inference}
\subsubsection{Prior and Posterior distribution}
\subsubsection{Posterior conjugation}
Conjugation of binomial distribution}
Lemma: binomial distribution is conjugated to beta distribution
Conjuga

Conjugation of multinomial distribution
% multinomial distribution dirichlet distribution categorical distribution
Lemma: multinomial is conjugated to dirichlet distribution

Summary t
Conjugation tables:
\subsubsection{Markov Chain Monte Carlo method}
\sub


\section{Modeling of bees colony}
\subsection{Biological description}
We consider the collective action of a bee colony. Each bee in a colony could
possibly sting after observing a threat in the surrounding environment, and warn
other bees by releasing pheromone. By sensing the pheromone released in the
environment, other bees in the colony may also sting. Since stinging leads to
the termination of an individual bee, it reduces the total defense capability as
well. We studies how the actions of a bee changes with regarding to its
surrounding the environment.\\
There are 3 assumptions on the system:
\begin{enumerate}
\item Each bee release an unit amount of pheromone immediately after stinging.
\item A bee dies after stinging and releasing pheromone. In the other words, no
  bee can sting more than once.
\item Stinging behaviour only depends on the concentration of pheromone in the
  environment.
\end{enumerate}
Given $N$ bees, let $s:\{1,\ldots,N\}\rightarrow[0,1]$ be the probability of
each bee sting given $i$ units of pheromone in the environment.\\

The question we are concerning about is that, given a system, how many bees
would sting in the steady state. Formally $Y$ is a discrete random variable
indicates the number of dead bees at steady state, we want to find the
probability distribution of $Y$ Given $n$ bees in an isolated box. After
stinging, each bee release unit $\Delta$ amount of pheromone and then dies. We
denote that
\begin{itemize}
\item $s(i)$ is the probability of a bee sting at a pheromone level $i\Delta
  (i\in \mathbb{N})$
\item At pheromone 
\end{itemize}

Regarding the way we observe the deadbees, there are two types of experiment
that will be described
\begin{itemize}
\item Synchronous
\item Asynchronous
\end{itemize}

\subsection{Synchronous and asynchronous system}


\subsection{Markov chain modeling of synchronous system}

\subsection{Frequentist approach}
Statistical method for parameter inference is presented in \cite{hajnal2019data}
Input: distribution on steady state of number of dead bees Output:

\section{Implementation and results}
\subsection{Data synthesize}
Data synthesis for each model is to synthesize the probabitity of being in each
BSCCs given a fixed set of parameter(\textit{true parameters})conducted as following:
. For each DTMC


\subsection{Experiment A: Bayesian inference with conjugation}
This experiment is designed to show the convergence to true steady state
distribution. 

\subsection{Experiment B: Bayesian inference without conjugation}
This experiment is designed

\subsection{Performance}
\textbf{Complexity: } For the first experiment, the complexity of updating steady state distribution
is linear to the number of BSCCs. However, for the second experiment, the
complexity grows faster.\\



\section{Conclusion}


% briefly introduction of data based method for parameter inference






\section{Discussion and further development}



\section{Introduction}
In this progress report, the following points are being presented:
\begin{enumerate}
\item Description of the system
\item Probability distribution of the number of stinging bees
\item Markov chain formalization
\item Parameter inference.
\end{enumerate}

\section{Formal description}

\subsection{Fully asynchronous experiment}
In asynchronous experiment we assume that there is almost improbable for two
bees to sting at exactly the same time, and any bees release pheromone
immediately after its death. By that observation we can assume that each bee
sting at different level of pheromone.\\

\textbf{Lemma 1} In \textit{asynchronous model}, the probability of seeing $j$
dead bees at the steady state is
\begin{align}
  P(Y=j) = {n\choose j}s(0)s(1)\ldots s(j-1)(1 - s(j))^{n-j} 
\end{align}

\textbf{Proof:} Since under asynchrnous assumption, we assume that at each
pheromone level there is at most one bee sting. Thus, $j$ stinging bees must do
so under different concentration of pheromone. Also under asynchronous update,
at steady state there are at most $j$ amount of pheromone diffused in the
environment. The other bees does not sting at any pheromone level from $0$ to
$j$, thus we can see all of them as do not stinging at the highest level $j$.\\
Selecting $j$ bees from $N$, each of them sting under pheromone level $0$ to
$j$, then the other $n-j$ bees does not sting under pheromone $j$, we deliver
Lemma 1 directly. \QED


\subsection{Fully synchronous experiment}
In \textit{fully synchronous} experiment we assume that the number of stinging
bees is only counted after a fixed amount of time, so that without loss of
generality we can assume the pheromone diffuse almost immediately among the bee
colony, and each bee decide to sting or not to sting immediately after sensing
the pheromone concentration.\\
Under the synchronous assumption, the Lemma 1 does not hold anymore. It is due
to the fact that it is possible to have more than one bee sting given the same
concentration of pheromone in the environment. Thus, we would like to construct
the probability distribution in a different way.\\
In order to construct a formula for fully synchronous setup, we use the
following observations:
\begin{enumerate}
\item if there are $k$ bees stinging, the amount of pheromone in the environment
  is $k$
\item there are $k$ intervals of stinging corresponding to pheromone
  amount of $k$, namely $(s(0),s(1)),\ldots,(s(k-1),s(k))$
\item placing bees into intervals must be consecutive, namely there exist no empty
  interval $(s(i),s(j))$ such that $(s(i-2),s(i-1))$ and $(s(j+1),s(j+2))$ are
  non empty.
\end{enumerate}
Based on these observations, we may propose to use the following counting
scheme for $k$ bees stinging: \textit{How many ways to place $k$ bees into
  consecutive intervals starting from $(s(0),s(1))$?}.\\
For a consecutive interval of size $l$, the number of ways to place $k$ bees
into the interval such that there is no empty interval is the Stirling number of
the Second type, namely:
\begin{align*}
  \stirlingii{k}{l} = \frac{1}{l!}\sum_{i=0}^{l}(-1)^i\binom{l}{i} (l - i)^k
\end{align*}
However, Stirling number of the Second kind assumes does not retain the
information of which the partition (interval) labels, which we need to calculate
the probability. We need to embed the information into our calculation, so that
the probability of $k$ bees stinging within $l$ intervals is 
\begin{align*}
  \frac{1}{l!}\sum_{i=0}^{l}(-1)^i\binom{l}{i} (l - i)^k (1-s(i))^l
\end{align*}
\textit{Huy: I am not sure about this step, namely how I add the information of
  the interval labels}
By that we calculate the probability that given $n$ bees, $k$ bees among
them stinging:
\begin{align*}
  P(Y=k) = \binom{n}{k} \sum_{i=0}^{k} \frac{1}{i!}\sum_{j=0}^{i}(-1)^j\binom{i}{j} (k - j)^i (1-s(i))^j
\end{align*}

However, this construction does not leads to the correct answer, e.g $(1-s_0)^2$
for $P(Y=0)$.....


\section{System modeling using Markov Chain}


\subsubsection{Problem declaration}
\subsubsection{Method description}

\printbibliography

\end{document}


