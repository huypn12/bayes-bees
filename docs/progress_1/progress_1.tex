\documentclass[12pt]{article}

\usepackage[utf8]{inputenc}

% graphics
\usepackage{graphicx}
\usepackage[pdf]{graphviz}

% math packages
\usepackage{amsmath}
\usepackage{amsfonts}
\usepackage{amssymb}

% citation
\usepackage{biblatex}
\addbibresource{progress_1.bib}

% title
\title{Master Project on MCSS\\ Progress Report 1}
\author{Huy Phung\\University of Konstanz}
\date{\today}

\setlength{\headheight}{23pt}
\setlength{\parindent}{0.0in}
\setlength{\parskip}{0.0in}


\begin{document}
\maketitle

\section{Introduction}
We consider the collective action of a bee colony. Each bee in a colony could
possibly sting after observing a threat in the surrounding environment, and warn
other bees by releasing pheromone. By sensing the pheromone released in the
environment, other bees in the colony may also sting. Since stinging leads to
the termination of an individual bee, it reduces the total defense capability as
well. We studies how the actions of a bee changes with regarding to its
surrounding the environment.\\
In this progress report, the following points are being presented:
\begin{enumerate}
\item Description of the system
\item Probability distribution of the number of stinging bees
\item Markov chain formalization
\item Parameter inference.
\end{enumerate}

\section{Formal description}
There are 3 assumptions on the system:
\begin{enumerate}
\item Each bee release an unit amount of pheromone immediately after stinging.
\item A bee dies after stinging and releasing pheromone. In the other words, no
  bee can sting more than once.
\item Stinging behaviour only depends on the concentration of pheromone in the
  environment.
\end{enumerate}
Given $N$ bees, let $s:\{1,\ldots,N\}\rightarrow[0,1]$ be the probability of
each bee sting given $i$ units of pheromone in the environment. \textit{Steady
  state}

The question we are concerning about is that, given a system, how many bees
would sting in the steady state. Formally $Y$ is a discrete random variable
indicates the number of dead bees at steady state, we want to find the
probability distribution of $Y$
Given $n$ bees in an isolated box. After stinging, each bee release unit
$\Delta$ amount of pheromone and then dies. We denote that
\begin{itemize}
\item $s(i)$ is the probability of a bee sting at a pheromone level $i\Delta
  (i\in \mathbb{N})$
\end{itemize}

Regarding the way we observe the deadbees, there are two types of experiment
that will be described
\begin{itemize}
\item Synchronous
\item Asynchronous
\end{itemize}

\subsection{Fully asynchronous experiment}
In asynchronous experiment we assume that there is almost improbable for
two bees to sting at exactly the same time, and any bees release pheromone
immediately after its death. By that observation we can assume that each bee
sting at different level of pheromone.\\

\textbf{Lemma 1} In \textit{asynchronous model}, the probability of seeing $j$
dead bees at the steady state is
\begin{align}
  P(Y=j) = {n\choose j}s(0)s(1)\ldots s(j-1)(1 - s(j))^{n-j} 
\end{align}

\textbf{Proof:} Since under asynchrnous assumption, we assume that at each
pheromone level there is at most one bee sting. Thus, $j$ stinging bees must do
so under different concentration of pheromone. Also under asynchronous update,
at steady state there are at most $j$ amount of pheromone diffused in the
environment. The other bees does not sting at any pheromone level from $0$ to
$j$, thus we can see all of them as do not stinging at the highest level $j$.\\
Selecting $j$ bees from $N$, each of them sting under pheromone level $0$ to
$j$, then the other $n-j$ bees does not sting under pheromone $j$, we deliver
Lemma 1 directly.\\


\subsection{Fully synchronous experiment}
In \textit{fully synchronous} experiment we assume that the number of
stinging bees is only counted after a fixed amount of time, so that without loss
of generality we can assume tha.pheromone diffuse
almost immediately among the bee colony, and each bee decide to sting or not to
sting immediately after sensing the pheromone concentration.\\
Under the synchronous assumption, the Lemma 1 does not hold anymore. It is due
to the fact that it is possible to have more than one bee sting given the same
concentration of pheromone in the environment. Thus, we would like to construct
the probability distribution in a different way.
\textbf{Lemma 2}


\section{System modeling using Markov Chain}


\subsubsection{Discrete time Markov Chain}
A \textit{Discrete Time Markov Chain} is a tuple, where
\begin{itemize}
\item a
\item b
  
\end{itemize}

\textit{Markov property}

% Why do we consider Markov property on modeling the bees?

\subsubsection{State spaces}

\subsubsection{Transition probabilities}

\subsubsection{}


\section{Modeling with Markov chain}
\subsection{Parameterized DTMC}


parameter inference problem is defined as \cite{katoen2016probabilistic}

\subsection{Statistical parameter inference}
\subsubsection{Definition}
Statistical method for parameter inference is presented in \cite{hajnal2019data}
Input: distribution on steady state of number of dead bees Output:

\subsubsection{Method description}
Observing

% briefly introduction of data based method for parameter inference

\section{Bayesian inference}
% what is a support
\subsection{Bayesian parameter inference}
Bayesian Formula:
\begin{align*}
  P(H|D) = \frac{P(D|H)P(H)}{P(D)}
\end{align*}


\textit{Prior distribution}


\textit{Posterior distribution}


\textit{Marginal distribution}



\subsection{Posterior conjugation}
% Problem: why do we need conjugation, which property of conjugation are we
% interested in? tractability?


\subsubsection{Conjugation of binary distribution}
Lemma:
\subsubsection{Conjugation of multinomial distribution}
% multinomial distribution
% dirichlet distribution
% categorical distribution



\subsubsection{Problem declaration}
\subsubsection{Method description}

\printbibliography

\end{document}


