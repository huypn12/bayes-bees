\documentclass[notitlepage]{report}

\usepackage[utf8]{inputenc}
\usepackage{titling}
\usepackage{graphicx}
\usepackage{amsmath}
\usepackage{amsfonts}
\usepackage{amssymb}
\usepackage{biblatex}
\addbibresource{proposal.bib}

\title{Master Project Proposal}
\author{Huy Phung\\University of Konstanz}
\date{February 2020}

\begin{document}
\maketitle

\section*{Abstract}
  We consider the collective action of a bee colony. Each bee in a colony could
  possibly sting after observing a threat in the surrounding
  environment, and warn other bees by releasing {pheromone. By sensing
  the pheromone released in the environment, other bees in the colony may also
  sting. Since stinging leads to the termination of an individual bee, it
  reduces the total defense capability as well. We studies how the actions of a
  bee changes with regarding to its surrounding the environment.

\section*{Introduction}
The bee colony can be considered as agents (bees) interact with each other in a
closed environment with the appearance of a factor \textit{pheromone}. Each
agent in the colony observe the following two factors of the environment: (1)
amount of pheromone and (2) number of other agents. Afterward, the agent has
probability to commit an action, namely \textit{sting}. The agent is eliminated
from environment after stinging. Under these assumption, we can see that a bee
colony lose defense capability as its individuals become more aggresive. This
project try to find out how environment affect the collective defense behaviour
of a bee colony by parametric probabilistic modeling and adjust the model
parameters by the data obtained from controlled experiments with bees.

\section*{Project Description}
First, we model the bee colony. Assume that each agent in the environment decide
its next action based only on the current observation, we use Markov Chain to
model the system. Under the assumption that the observation is conducted in
discrete time, the model is \textit{DTMC}. However, \textit{CTMC} should be used
since it can express the time properties of the colony more precisely. In order
to model the collective behaviour of bees, we use \textit{synchronous product},
which is a method to composite Markov Chains with an assumption that all
transitions are taken together. If the order of each agents' state is important,
we use \textit{asynchronous product}, where each transition is taken separately,
which results in a larger set of states.\\
The next problem we cope is \textit{parameters synthesis}. Since the transition
probability on Markov is not known apriori, we use symbolic to represent them,
then synthesize it with the data gained from experiments
\cite{katoen2016probabilistic}. Currently in \cite{10.1007/978-3-030-28042-0_10}
the authors are using SMT solvers to find safe intervals for model parameters
with regarding to the experiments data of probability to be in certain states.
We propose to apply Bayesian method \cite{10.1007/978-3-319-43425-4_3} to infer
the parameters for DTMC and CTMC. The authors also uses Metropolis Hasting to
calculate the safe region. We also want to compare the results from Bayesian
inference to Metropolis Hasting.

\section*{Expected Results}
From \textit{probabilistic model checking} point, this project should (1) be
capable of modeling the bee colony, and (2) infer the parameter from experiments
data and possibly (3) checking properties of the model.\\
Also, comparison must be conducted to evaluate the effectiveness of Bayesian
inference in parameter synthesis to the current statistical method

\section*{References}
\printbibliography

\end{document}
